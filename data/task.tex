% !Mode:: "TeX:UTF-8"
\pagestyle{empty}
\vspace*{1ex}
\myoverlay%为当前页添加边框
{\cusong\sihao\hspace*{-2em} 一、设计内容}
     \begin{enumerate}
     \item 准备了解有关导航以及导航系统的特点、应用前景。
     \item 准备对导航系统中常用的惯导器件—陀螺仪进行说明。
     \item 准备对提高导航信号测量精度所用到的滤波技术进行软件仿真。
     \item 准备分析比较各种滤波信号,并得出相应结论。
     \end{enumerate}

\myhline%划横线
{\cusong\sihao 二、设计原始资料}
     \begin{enumerate}
     \item 导航系统的相关资料
     \item 惯导器件基础教程及有关书籍
     \item 卡尔曼滤波、神经网络相关资料
     \item \Matlab{}软件教程及相关资料
     \end{enumerate}

\myhline%划横线
{\cusong\sihao 三、设计完成后提交的文件和图表}
     \begin{enumerate}
     \item 计算说明书部分
     \begin{enumerate}
     \item 卡尔曼滤波、扩展卡尔曼滤波以及无迹卡尔曼滤波对比仿真图
     \item 各种滤波算法应用范围对比表
     \end{enumerate}
     \item 图纸部分
     \begin{enumerate}
     \item 几种惯性导航系统原理框图
     \item 改进卡尔曼滤波效果仿真图
     \item 卡尔曼滤波、扩展卡尔曼滤波以及无迹卡尔曼滤波效果仿真比较图
     \end{enumerate}
     \end{enumerate}

\myhline
{\cusong\sihao 四、毕业设计进程安排}

          \vspace{2em}
          {\heiti 序号\hspace{5em}设计各阶段名称\hspace{9em}日期(教学周)}

          {\hspace{0.5em}1\hspace{3em}阅读相关书籍,完成开题报告\hspace{4em}3月1 日--3月14日(第1--2周)}

          {\hspace{0.5em}2\hspace{3em}准备了解常见导航系统的相关知识\hspace{2em}3月15 日--3月28日(第3--4周)}

           {\hspace{0.5em}3\hspace{3em}研究导航系统中较常见的惯导器件\hspace{2em}3月29日--4月11日(第5--6周)}

           {\hspace{0.5em}4\hspace{3em}学习导航系统中几种常用滤波技术\hspace{2em}4月12日--5月9日(第7--10周)}

           {\hspace{0.5em}5\hspace{3em}准备对滤波算法进行仿真比较\hspace{4em}5月10日--5月16日(第11周)}

           {\hspace{0.5em}6\hspace{3em}撰写毕业论文\hspace{11em}5月17日--6月10日(第12--14周)}

\myhline
{\cusong\sihao 五、主要参考资料}
\begin{bibList}
\item 于国强. 导航与定位. 北京:国防工业出版社.2000.2:1.
\item 邓正隆. 惯性导航原理.哈尔滨:哈尔滨工业大学出版社.1994.1:1-2,2-5.
\item 袁建平. 卫星导航原理与应用.中国宇航出版社.2003 第一版:23-24.
\item 秦永元. 惯性导航.北京:中国科学出版社.2006:1-5.
\item 陈开权. 惯性导航的理论基础[J]. 北京: 水雷战与舰船防护,2003(1):76-90.
\item 付梦印,邓志红,张继伟.Kalman 滤波理论及其在导航中的应用.北京:科学出版社.2003.
\end{bibList}

%%%%%%%%%%%%%%%%%%%  长安大学毕业设计(论文)开题报告表  %%%%%%%%%%%%%%%%%%%%%%%
\newpage
\markboth{长安大学毕业设计(论文)开题报告表}{长安大学毕业设计(论文)开题报告表}
\pdfbookmark[0]{长安大学毕业设计(论文)开题报告表}{creport}
\begin{center}
\hei\sanhao{长安大学毕业设计(论文)开题报告表}
\end{center}
\begin{table}[h]
  \centering\xiaosi
  \begin{tabularx}{\textwidth}{|c|p{0.15\textwidth}|c|X|c|X|}
     \hline
     课题名称 & \multicolumn{5}{c|}{TeX惯性器件精度提高实现方法} \\ \hline
     课题来源 & \centering {自选课题} & 课题类型 & \centering {工程设计} & 指导教师 & \centering {CTeX} \tabularnewline \hline
     学生姓名 & \centering {Tsingber  Lee} & 学\hspace*{24bp}号 & \centering {2403000001} & 专\hspace*{24bp}业 & \centering {电子信息工程} \tabularnewline \hline
   \end{tabularx}
      \centering\xiaosi
\end{table}
%%%%%%%%%%%%%%%%%画图区%%%%%%%%%%%%%%%%%
\setlength{\unitlength}{1mm}
\noindent\begin{picture}(0.1,0)
\multiput(0.1,0)(160.8,0){2}{\line(0,-1){188}}
\multiput(0.1,0)(160.8,0){2}{\line(0,1){9.3}}
\put(0.1,-188){\line(1,0){160.8}}
\end{picture}
%%%%%%%%%%%%%%%%%%%%%%%%%%%%%%%%%%
     {  {\heiti (一)课题意义:}

 由半导体工业中的微细加工技术与机械工业中的微型机械加工技术结合而产生并逐渐发展起来的微电子机械系统(MEMS: Microelectro Mechanical Sys2tem) 是微米、纳米电子学的重要领域,是一项极具发展前景的军民两用高技术,它的出现,将引发一场新的技术革命。

微电子机械系统涉及到微电子学、自动控制、光学、气动力学、流体力学和声学磁学等多种领域,可以说是一门多学科的综合技术。它研究的主要内容包括微型传感器、微型执行器和复杂的微系统。提高MEMS加速计传感器的精度具有重要意义。

{\noindent\heiti (二)国内外发展状况:}

(1)国外概况:

MEMS 技术的开发始于20 世纪60 年代,其迅速发展是在20世纪80年代末期,由MEMS技术的迅速发展,在1987年便决定把MEMS从IEEE国际微机器人与过程操作年会分开,单独召开年会。目前每年在美、日、欧三地轮回举行名为IEEE 国际MEMS年会(Microelectro Mechanical Systems Workshop ) 。

美国是研究开发MEMS最早的国家,早在上世纪60年代加利福尼亚大学和贝尔实验室就开始这方面的研究,曾开发出微型硅压力传感器,后又在20世纪70年代开发出硅片色谱仪、微型继电器。特别是在20世纪90年代初,加大利用“牺牲层”技术,制造出一台直径小于人发的超微静电电动机,曾在世界引起很大轰动,专家纷纷预测其广阔的应用前景。此后,美国超微机电系统的研究走上有序开发阶段,取得了一系列被专家称为具有划时代意义的成果。

1987年,美国加州大学Berekeley分校的范龙生等人在第四届国际固态传感器与执行器会议上,报道了用表面微机械加工技术制作的多晶硅齿轮,引起了世界各国科学家的注意,此后微电子机械系统工程成为人们关注的新兴学科。LIGA 技术是在20世纪80年代中期由德国Karlsruhe原子核研究中心发展起来的,包括同步辐射深度光刻、微电铸、微塑铸3个过程。
目前全世界研制生产MEMS有600多个单位,已研究出几百种产品,其中微电子传感器占大部分。


     (2)国内概况:

我国开展MEMS 研究始于20世纪80年代末,10多年来研究队伍迅速发展和扩大,目前已有40多个单位的50多个研究小组,在新原理微器件、通用微器件、新的工艺和测试技术以及初步应用等方面取得了显著的进展。

1995年国家科技部实施了攀登计划“微电子机械系统项目”(1995~1999 年) 。1999年“集成微光机电系统研究”项目通过了国家重点基础研究发展规划的立项建议。我国已开展了包括微型直升飞机、力平衡加速度传感器、力平衡真空传感器、微泵、微喷嘴、微电机、微电泳芯片、微流量计、硅电容式麦克风、分裂漏磁场传感器、集成压力传感器、微谐振器和微陀螺等许多微机械器件的研究和开发工作。研究出的硅电容式微麦克风是目前国际同类研究中灵敏度最高的;在国际上首次研制出包括片上转速检测电路的集成硅微静电电机;分裂漏磁场传感器、集成压力传感器、硅微静电电机工艺已取得三项美国专利。

我国的MEMS研究已经历了10多年的发展,虽然有一些器件和机构被研制出来,但目前整体仍处于基础性阶段,极少有实用的MEMS器件。与国际上的最大差别是在产业化推进方面尚未具备大批量生产的能力。为此,我们应该根据MEMS器件制造的特点,从工艺研究入手,对一些有一定研究基础、应用面广、市场前景好的MEMS产品,进行重点攻关,掌握成熟的制造工艺,为产业化推进铺平道路。

{\noindent\heiti (三)本课题的研究内容:}

MEMS惯性器件存在测量精度低、噪声大等缺点,需要采取一些必要的措施以提高其精度,除了优化机械结构设计,提高电子线路的性能,以及采取屏蔽外部电磁干扰措施之外,另一种有效的途径是:从应用的角度对MEMS惯性器件和微惯性测量单元(MIMU)进行误差分析及补偿,提高系统的测量精度。本课题在主要通过改进滤波器来提高其精度,内容有:

1.首先对导航技术中几个主要惯性器件进行初步介绍

2.对于导航技术中应用十分广泛的几种滤波技术进行了比较深入地研究

3.通过神经网络等技术对其进行初步改进

{\noindent\heiti (四)本课题的研究方法:}

首先介绍几种常见的导航系统,之后对应用最为广泛的惯性导航系统所使用的滤波技术进行了深入研究说明,并提出自己的改进方法。最后编写相应程序以测试方法的优缺点。

\myoverlay%为当前页添加边框

{\noindent\heiti (五)本课题的研究手段:}

阅读有关MEMS导航系统相关资料,对导航理论有一个基本的了解,在些基础上,对相关的滤波算法进行一定的学习与比较,了解算法的基本原理和思想。使用Matlab软件编程,对滤波器进行仿真对比。	

{\noindent\heiti (六)本课题的研究成果:}

本报告通过Matlab仿真比较得出最为有效、滤波效果最佳的滤波方法。

     {\noindent\heiti (七)任务完成的阶段安排及时间安排:}

(1)3月1日-3月14日(第1-2周)

阅读相关书籍,完成开题报告。

(2)3月15日-3月28日(第3-4周)

学习导航系统,对常见的导航系统有所了解。

(3)3月29日-4月11日(第5-6周)

研究导航系统中常用的惯导器件,对其中较为关键的惯导器件—陀螺仪有比较清楚地认识。

(4)4月12日- 5月9日(第7-10周)

对导航系统中涉及的滤波技术以及神经网络技术作深入学习,了解它们的特点。

(5)5月10日-5月16日(第11周)

对各种滤波算法进行仿真比较。

\myoverlay%为当前页添加边框
\myoverlay%为当前页添加边框
(6)5月17日- 6月8日(第12-14周)

撰写毕业设计论文

(7)6月9日-6月17日(第15周)

提交论文评审与答辩

     {\heiti (八)任务所具备的条件因素:}\myoverlay

(1)对导航及导航技术有初步了解。

(2)了解并清楚常见导航系统之间的优缺点,知道它们的应用前景。

(3)熟悉导航滤波技术中各种滤波算法的原理及特点。

(4)能够比较熟练地运用Matlab仿真软件。

\myhline

指导教师意见及建议: \\
     \vfill
     {\hfill 指导教师签名:\hspace*{5cm}}

     \vspace{1em}
     {\hfill 年\hspace*{0.7cm}月\hspace*{0.7cm}日\hspace*{5cm}}\vspace*{1cm}
\thispagestyle{empty}