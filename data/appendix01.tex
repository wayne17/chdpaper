% TeX
%\chapter{部分仿真程序}
\chapter{部分仿真程序}\label{ch:fz}
\lstset{% general command to set parameter(s)
basicstyle=\ttfamily\small, % print whole listing small
%keywordstyle=\color{black}\bfseries,
% \underbar underlined bold black keywords
identifierstyle=, % nothing happens
breaklines=true,
commentstyle=\itshape\color{gray},
%stringstyle=\ttfamily, % typewriter type for strings
showstringspaces=false, % no special string spaces
stepnumber=2}
\begin{lstlisting}[language=Matlab,caption={最小二乘法获取系统传递函数}]
%%
% 输入参数:
% x  ---  系统输入阵列
% y  ---  系统输出阵列
% N  ---  系统传递函数的阶数
% 函数输出:
% 系统 Z 传递函数分子与分母的系数,即: h(z) = numd(z) / dend(z)

%%

function [numd, dend] = LeastSquare(x, y, N)
count = length(y);
M = count - 1;
ai = zeros(N*2, M);
for i=1:N
    ai(i, i:M) = x(1:(count-i));
    ai(i+N, i:M) = -y(1:(count-i));
end
bi = y(2:(count));
xd = (inv(ai*ai')*ai*bi)';
numd = [0 xd(1:N)];
dend = [1 xd(N+1: N+N)];
\end{lstlisting}
\begin{lstlisting}[language=Matlab,caption={拥有精确模型的卡尔曼滤波}]
clc
clear
close all

Ts = 1;               %采样时间
s = tf('s');
sysc = 100/(60*s +1);   %真实系统的传递函数
sysd = c2d(sysc, Ts);

t = 0:Ts:(Ts*300);
u = ones(1, length(t));   %系统输入
ys = lsim(sysd, u, t, 0); %系统输出
ys = ys + 10*rand(size(ys)); %测量量 = 系统输出 + 噪声

%% 就是这个地方在基于新息的那篇论文里用的函数
[numd, dend] = LeastSquare(u, ys, 3);   %% 最小二乘法获取预测系统模型
[A, B, C, D] = tf2ss(numd, dend);       % 变换到状态空间形式
Len = length(u);
N = length(A);        % 系统维数

yout = zeros(Len, 1);  %滤波器输出
Xh = zeros(N, 1);      %状态变量
P = eye(N);
Q = 0.02 * eye(N);     %系统噪声
R = 50;                %测量噪声
for i = 1 : Len
    Xh = A*Xh + B*u(i);
    P = A*P*A' + Q;
    K = P*C'* inv(C*P*C' + R);
    Xh = Xh + K*(ys(i) - C*Xh);
    P = P - K*C*P;
    yout(i) = C*Xh;
end

plot(t, ys, t, yout);
\end{lstlisting} 