% !Mode:: "TeX:UTF-8"
%%% Local Variables:
%%% mode: latex
%%% TeX-master: "../main"
%%% End:

\begin{conclusion}
该部分主要包括两部分:“结论”和“展望”。结论是理论分析和实验结果的逻辑发
展,是整篇论文的归宿。结论是在理论分析、试验结果的基础上,经过分析、推理、判
断、归纳的过程而形成的总观点。结论必须完整、准确、鲜明、并突出与前人不同的新
见解。总结部分还应说明论文中的创新点内容。创新点应该以分条列举的形式进行提
出。展望是对该研究课题存在的不足和有待改进的说明,是对未来研究的一种期待。该
部分的字数应不少于半页\upcite{ELIDRISSI94,
  MELLINGER96, SHELL02}。
这里测试下myList环境:
\begin{myList}
        \item 封面;
        \item 毕业论文(设计)任务书;
        \item 开题报告;
        \item 中英文摘要及关键词
        \item  目录;
        \item  正文;
        \item  致谢
        \item  参考文献或资料;
        \item  附录(包括计算程序及说明、过长的公式推导等);
        \item  附件(包括图纸、外文文献\upcite{ELIDRISSI94}译文等);
\end{myList}

请直接单面打印PDF文件,空白页已经按要求留出。打印时,缩放页面的选项设
为“无”,否则页面会缩小。
\end{conclusion}